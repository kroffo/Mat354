\documentclass{scrartcl}
\usepackage{amsmath,amssymb,commath,graphicx,enumerate,listings}
\setkomafont{disposition}{\normalfont\bfseries}

\title{Mat 354}
\subtitle{Homework 10}
\author{Kenny Roffo}
\date{Due October 30, 2015}

\begin{document}
\maketitle

All calculations were done in R.

\begin{enumerate}

\item \textbf{2 Bounds}\\

A random variable has expected value 8.8 and variance 1.76. Obtain the following.

\begin{enumerate}[a)]
\item An upper bound for the probability of an outcome at least 5 from the mean.\\

Using Chebychev's Inequality:
\begin{align*}
  &P(|Y-\mu| \ge \sigma k) \le \frac{1}{k^2}\\
  \implies &P(|Y-8.8| \ge \sqrt{1.76}k)  \le \frac{1}{k^2}\\
\end{align*}
To find the upper bound so the outcome is at least 5 from the mean, $\sigma k$ must be at least 5. Thus $k=\frac{5}{\sqrt{1.76}}=3.7689$. Thus the upper bound is $$\frac{1}{k^2} = \frac{1}{14.2045} = 0.0704 $$

\item An interval (centered at the mean) for which the probability of an outcome within that interval is at least 0.6. Please be mathematically precise in stating what your interval is.\\

We know Chebychev's Inequality: $$P(|Y-\mu| < \sigma k) > 1 - \frac{1}{k^2}$$ For the probability to be at least 0.6, $1 - \frac{1}{k^2} \ge 0.6$. Thus implies $k \ge 1.5811$, so the probability is at least 0.6 for $$P(|Y-8.8| < \sqrt{1.76}(1.5811))$$ thus the interval within which the probability is at least 0.6 is $$(-\sqrt{1.76}(1.5811),\sqrt{1.76}(1.5811)) = (-2.0976,2.0976)$$

\end{enumerate}

\item \textbf{Max Again}

Start with a SRS of 4 from the population consisting of the values 1, 2, ... 10. We’ve seen (review old HW) that the probability function for the random variable $Y$ that is the maximum is $$p(y) = \frac{{y \choose 4}-{y-1 \choose 4}}{{10 \choose 4}}\text{\hspace{1in}} y\in\mathbb{Z}\cup[4,10]$$ to get this into R:
\begin{lstlisting}[language=R]
  > y <- seq(4,10)
  > p <- (choose(y,4) - choose(y-1,4))/choose(10,4)
\end{lstlisting}
\begin{enumerate}[a)]
\item Determine the mean and variance of this random variable.\\
The mean:
\begin{align*}
  E(Y) &= \sum_{y=4}^{10}y\cdot p(y)\\
  &= 8.8
\end{align*}

The variance:
\begin{align*}
  V(Y) &= E(Y^2) - E(Y)^2\\
  &= \sum_{y=4}^{10}\left[y^2\cdot p(y)\right] - \sum_{y=4}^{10}\left[y\cdot p(y)\right]^2\\
  &= 1.76
\end{align*}

\item Determine the probability of an outcome more than 5 from the mean.

Since everything 5 from the mean is less than 4 and greater than 10, this probability is just 0.\\

\item What’s the probability of an outcome falling within the interval solving part b of 2 Bounds.

Since $\mu$ and $\sigma$ are the same as the first question, the probability is 0.6 that $Y$ is in the interval from part b of question 1.\\

\item The mode of the probability distribution is the value $y_0$ that has the largest probability: $y_0$ is mode if $p(y_0) \ge p(y)$ for all $y$. (Some distributions have multiple modes – where two or more values “tie” for most probable. What’s the mode of this distribution?

The mode of this distribution is 10.\\

Now consider a SRS of size $n$ from the population consisting of the values $1, 2, ..., N$ where $N \ge n > 1$. Again $Y$ represents the maximum value.
\item State the probability function for Y.

The probability function is given by: $$p(y) = \frac{{y \choose n}-{y-1 \choose n}}{{N \choose n}}\text{\hspace{1in}} y\in\mathbb{Z}\cup[n,N]$$

\item Assume $y$ is an integer satisfying $(n + 1) \le y \le N$ (that is: $y$ is a “sensible” value). Show that $p(y) > p(y-1)$. (One good way considers (starts on the left with) $p(y)/p(y-1)$ and simplifies step by step to reveal a quantity $> 1$ – which would conclude your demonstration on the right.) Hint: ${y \choose n} - {y-1 \choose n}={y-1 \choose n-1}$.

\begin{align*}
  \frac{p(y)}{p(y-1)} &= \frac{\frac{{y \choose n}-{y-1 \choose n}}{{N \choose n}}}{\frac{{y-1 \choose n}-{y-2 \choose n}}{{N \choose n}}}\\
  &= \frac{{y \choose n}-{y-1 \choose n}}{{y-1 \choose n}-{y-2 \choose n}}\\
  &= \frac{{y-1 \choose n-1}}{{y-2 \choose n-1}}
\end{align*}
But this is clearly greater than 1. The are obviously less ways to choose 3 rocks from 4 rocks than there are from 5 rocks. Thus we have
\begin{align*}
  &\frac{p(y)}{p(y-1)} > 1\\
  \implies &p(y) > p(y-1)
\end{align*}

\item What is the mode of the distribution for Y?
  Part f implies that the mode of the distribution for $Y$ is $N$.\\

\item Determine an expression for the expected value of $Y$. Please explain thoroughly the process you use in obtaining this result (including what advice or guidance you may get from others). DO NOT take ownership of methods in mathematics that you cannot reasonably generalize. (If you have a candidate expression and take $n = 4$ and $N = 10$, you should get $E(Y) = 8.8$; taking n = N you must obtain $E(Y) = N$, right?)

The expected value is:
\begin{align*}
  E(Y) &= \sum_{y=n}^Nyp(y)\\
  &= \sum_{y=n}^Ny\frac{{y-1 \choose n-1}}{{N \choose n}}\\
  &= \frac{1}{{N \choose n}} \sum_{y=n}^Ny\frac{(y-1)!}{(y-n-2)!(n-1)!}\\
  &= \frac{1}{{N \choose n}(n-1)!} \sum_{y=n}^N\frac{y!}{(y-n-2)!}\\
  &= \frac{(N-n)!n!}{N!(n-1)!} \sum_{i=0}^N\frac{(n+i)!}{(i-2)!}\\
  &= \frac{(N-n)!n}{N!} \sum_{i=0}^{N-n}\frac{(n+i)!}{(i-2)!}\\
\end{align*}

\end{enumerate}

\item \textbf{Lack of Lack of Memory}

Suppose a and b are positive integers. For any Geometric random variable $Y$, $$P(Y > a + b | Y > a) = P(Y > b)$$
(This is the “memoryless” or “lack of memory” property of this particular random variable. See Exercise 3.71 where part c tells us that this is “obvious”.) Show that this property does not apply to a Binomial random variable where $0 < p < 1$. (This demonstrates also that the lack of memory property is not simply a consequence of the independent trials.) The simpler and less computational your example is, the more bonus points you will receive.

Let $p = .6, a = 8, b = 4$ and $n = 10$. Then $$P(Y>a+b)=P(Y>12)=0$$, but obviously $P(Y>b)>0$, so in general the equality does not hold in a binomial setting.

\end{enumerate}
\end{document}

