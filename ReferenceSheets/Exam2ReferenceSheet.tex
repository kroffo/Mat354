\documentclass{scrartcl}
\usepackage{amsmath,amssymb,commath,graphicx}
\usepackage[margin=0.5in]{geometry}
\usepackage[T1]{fontenc}
\setkomafont{disposition}{\normalfont\bfseries}

\title{Mat 354 - Exam 2 Reference Sheet}

\begin{document}
\begin{Large}\textbf{Discrete Random Variables}\end{Large}\\

\textbf{Binomial} - Fixed number of trials, success or failure. $Y$ is number of trials until first success.
$$P(y) = {n \choose y}p^y(1-p)^{n-y} \text{\hspace{0.5in}} F(y) = \text{pbinom($y$,size,prob)} \text{\hspace{0.5in}} E(Y) = np \text{\hspace{0.5in}} V(Y) = np(1-p)$$

\textbf{Geometric} - Do trials until success. $Y$ is the number of of the first success trial.
$$P(y) = (1-p)^{y-1}p \text{\hspace{0.5in}} F(y) = 1 - (1-p)^y \text{\hspace{0.5in}} E(Y) = \frac{1}{p} \text{\hspace{0.5in}} V(Y) = \frac{1-p}{p^2}$$

\textbf{Hypergeometric} - $N$ objects. Simple Random Sample of $n$ objects. $r$ are type A, $N-r$ are type B. $Y$ is the number of type A in the $n$.
$$P(y) = \frac{{r \choose y}{N-r \choose n-y}}{{N \choose n}} \text{\hspace{0.5in}} F(y) = \text{phyper($y,r,N-r,n$)} \text{\hspace{0.5in}} E(Y) = \frac{1}{p} \text{\hspace{0.5in}} V(Y) = \frac{1-p}{p^2}$$

\textbf{Poisson} - Defined for $\lambda > 0$. The limit of the binomial distribution as $n\rightarrow\infty$ and $p \rightarrow0$. Think of a box of pollutant particles, slicing it into small slices to check if a particle is in each slice.
$$P(y) = \frac{e^{-\lambda}\lambda^y}{y!} \text{\hspace{0.5in}} F(y) = \text{ppois($y,r,N-r,n$)} \text{\hspace{0.5in}} E(Y) = \lambda \text{\hspace{0.5in}} V(Y) = \lambda$$

\textbf{Chebychev's Inequality}:  \hspace{0.5in} $1 - \frac{1}{k^2} < P(|Y-\mu|<\sigma k)$ \hspace{0.5in}$\frac{1}{k^2} \ge P(|Y-\mu|\ge\sigma k)$\\\ \\
\begin{Large}\textbf{Continuous Random Variables}\end{Large}\\

\textbf{Quantiles} - The $p^{th}$ quantile ($100p^{th}$ percentile) of $F(Y)$ is $\phi_p$ such that $F(\phi_p)=p$.\\

\textbf{Density Function} - $f(y) = \od{}{y}F(y)$ \hspace{0.5in} $E(Y) = \int_{-\infty}^{\infty}yf(y)dy$ \hspace{0.5in} $V(Y) = E(Y^2) - E(Y)^2$\\

\textbf{Exponential} - $F(y) = 1 - e^{-\frac{y}{\beta}}$, $\beta = \frac{1}{\lambda}$\\

\textbf{Normal} - Density function: $\phi(z) = \frac{1}{\sqrt{2\pi}}e^{-\frac{z^2}{2}}$ is an even function $\left(\phi(-a)=\phi(a)\right)$ (This is the standard normal)\\
Use qnorm to find quantiles. pnorm integrates the density function to give $F(y)$
Define for constants $a$ and $b$ $$f(y) = \frac{1}{\sqrt{2\pi}b}e^{-\frac{(y-a)^2}{2b^2}}$$ If a random variable $Y$ has this density, then the random variable $Z=\frac{Y-a}{b}$ is standard normal.
$Y$ has normal distribution with mean $\mu$ and variance $\sigma^2>0$ if $$f(y) = \frac{1}{\sqrt{2\pi}\sigma}e^{-\frac{(y-\mu)^2}{2\sigma^2}}$$ Probabilities are found by rescaling.\\

\textbf{Gamma Function} - $\Gamma(\alpha) = \int_0^\infty x^{\alpha-1}e^{-x}dx$ \hspace{0.5in} For an integer $k$, $\Gamma(k) = (k-1)!$
$$\text{For } \alpha>0, \beta>0 \text{\hspace{0.5in}} f(y) = \begin{cases} \frac{y^{\alpha-1}e^{-y/\beta}}{\Gamma(\alpha)\beta^\alpha} & y>0\\ 0 & y \le 0 \end{cases} \text{\hspace{0.5in}} E(Y) = \alpha\beta \text{\hspace{0.5in}} V(Y) = \alpha\beta^2$$
\end{document}
