\documentclass{scrartcl}
\usepackage{amsmath,amssymb,commath,graphicx,enumerate}
\setkomafont{disposition}{\normalfont\bfseries}

\title{Mat 354}
\subtitle{Homework 3}
\author{Kenny Roffo}
\date{Due September 16, 2015}

\begin{document}
\maketitle
\begin{enumerate}
\item A box of hard candy contains 19 white candies, 8 red, and 15 blue. If you select 6 pieces of candy randomly from the box, without replacement, give the probability that...

\begin{enumerate}[a)]

\item Exactly 3 of the candies are white:\\
$$ P = \frac{{19 \choose 3}{15 + 8 \choose 3}}{{42 \choose 6}} = 0.3271386$$

\item There are two candies of each color:\\
$$ P = \frac{{19 \choose 2}{8 \choose 2}{15 \choose 2}}{{42 \choose 6}} = 0.09583693$$

\item Every one of the sampled candies are either red or white:\\
$$ P = \frac{{19 + 8 \choose 6}}{{42 \choose 6}} = 0.05642815$$

\item Each color is represented:\\
$$ P = 1 - \left(\frac{{19 + 8 \choose 6}}{{42 \choose 6}} + \frac{{19 + 15 \choose 6}}{{42 \choose 6}} + \frac{{15 + 8 \choose 6}}{{42 \choose 6}}\right) = 0.6679504$$

\end{enumerate}
\pagebreak

\item You are meeting two people at the \emph{Gare du Nord} in Paris, one arriving from Amsterdam, the other arriving from Brussels. The trains are scheduled to arrive at the same time. Let $A$ and $B$ denote the events that the trains you are meeting are on time. The probability the Amsterdam train is on time is $P(A) = 0.93$. The Brussels train is on time with probability 0.89; the probability both trains are late is 0.05. Find the probability that...

\begin{enumerate}[a)]

\item At least one train is on time:\\
\begin{align*}
P(A \cup B) &= 1 - P(\bar{A}\cap\bar{B})\\
            &= 1 - 0.05\\
            &= 0.9500
\end{align*}

\item Both trains are on time:\\
\begin{align*}
P(A \cap B) &= 1 - P(\bar{A}\cup\bar{B})\\
            &= 1 - \left(P(\bar{A}) + P(\bar{B}) - P(\bar{A}\cap\bar{B})\right)\\
            &= 1 - \left(0.07 + 0.11 - 0.05\right)\\
            &= 0.8700
\end{align*}

\item Exactly one train is on time:\\
\begin{align*}
P(A \cup B - A \cap B) &= P(A \cup B) - P(A \cap B)\\
                       &= 0.95 - 0.87\\
                       &= 0.0800
\end{align*}

\item At least one train is late:\\
\begin{align*}
P(\bar{A \cap B}) &= 1 - P(A \cap B)\\
                  &= 1 - 0.87\\
                  &= 0.1300
\end{align*}

\item The Brussels train is on time given that the Amsterdam train is on time:\\
\begin{align*}
P(B|A) &= \frac{P(A \cap B)}{P(A)}\\
       &= \frac{0.87}{0.93}\\
       &= 0.9355
\end{align*}

\item Given the Brussels train is late, the Amsterdam train is on time:\\
\begin{align*}
P(A|\bar{B}) &= \frac{P(A\cap\bar{B})}{P(\bar{B})}\\
             &= \frac{P(A - A \cap B)}{P(\bar{B})}\\
             &= \frac{P(A)-P(A \cap B)}{P(\bar{B})}\\
             &= \frac{0.93-0.87}{1-0.89}\\
             &= \frac{0.06}{0.11}\\
             &= 0.5455
\end{align*}

\item Given the Brussels train is late, the Amsterdam train is late:\\
\begin{align*}
P(\bar{A}|\bar{B}) &= \frac{\bar{A}\cap\bar{B}}{P(\bar{B})}\\
                   &= \frac{P(\bar{A \cup B})}{P(\bar{B})}\\
                   &= \frac{1 - P(A \cup B)}{P(\bar{B})}\\
                   &= \frac{1 - 0.95}{1 - 0.89}\\
                   &= \frac{0.05}{0.11}\\
                   &= 0.4545
\end{align*}

\item The Brussels train is on time given that at least one of the trains is on time.\\
\begin{align*}
P(B|A \cup B) &= \frac{P(B \cap (A \cup B))}{P(A \cup B)}\\
              &= \frac{P(B)}{P(A \cup B)}\\
              &= \frac{0.89}{0.95}\\
              &= 0.9368
\end{align*}
\pagebreak

\item The Brussels train is on time given that exactly one of the trains is on time.\\
\begin{align*}
P(B|(A \cup B - A \cap B)) &= \frac{P(B \cap (A \cup B - A \cap B)}{P(A \cup B - A \cap B)}\\
                           &= \frac{P(B - A \cap B)}{P(A \cup B - A \cap B)}\\
                           &= \frac{P(B)-P(A \cap B)}{P(A \cup B - A \cap B)}\\
                           &= \frac{0.89 - 0.87}{0.08}\\
                           &= 0.2500
\end{align*}

\item The Brussels train is late given that at least one of the trains is late.\\
\begin{align*}
P(\bar{B}|\bar{A}\cup\bar{B}) &= \frac{P(\bar{B}\cap(\bar{A}\cup\bar{B})}{P(\bar{A}\cup\bar{B})}\\
                               &= \frac{P(\bar{B})}{P(\bar{A}\cup\bar{B})}\\
                               &= \frac{0.11}{0.13}\\
                               &= 0.8462
\end{align*}

\item $A$ and $B$ are independent events. Explain.\\
$$P(\text{A and B are independent events}) = 0$$
Since $P(A \cap B) = 0.87 \ne 0.8277 = P(A)P(B)$ we know that $A$ and $B$ are not independent.
\end{enumerate}

\item Some combinations:

\begin{enumerate}[a)]

\item Compute the following pairs of values:
\begin{enumerate}[i)]
\item $7 \choose 2$ and $7 \choose 5$
\begin{displaymath}
{7 \choose 2} = \frac{7!}{(7-2)!2!} = 21 = \frac{7!}{(7-5)!5!} = {7 \choose 5}
\end{displaymath}

\item $13 \choose 5$ and $13 \choose 8$
\begin{displaymath}
{13 \choose 5} = \frac{13!}{(13-5)!5!} = 1287 = \frac{13!}{(13-8)!8!} = {13 \choose 8}
\end{displaymath}
\end{enumerate}

\item What value $j$ not equal to 123456789 makes this true?
\begin{displaymath}
  {9876543210 \choose 123456789} = {9876543210 \choose j}
\end{displaymath}
The value $j = 9876543210-123456789 = 9753086421$

\item $k$ is a positive integer. What integer other than $k-1$ correctly fills in the first blank? Simplify the expression to correctly fill in the second blank.
\begin{displaymath}
  {k \choose k-1} = {k \choose \text{\underline{\ \ \ }}} = \text{\underline{\ \ \ \ \ \ \ \ }}
\end{displaymath}
The integer 1 will correctly fill in the first blank. The second blank is filled in by $k$, that is $${k \choose k-1}={k \choose 1}=k$$\\

The previous three parts of this problem all fall as a result that the task of choosing $k$ things out of a group of size $n$ is the same as choosing $n-k$ things to leave out, and taking the rest as your choice.

\item Determine exact values for the following:
\begin{enumerate}[i)]

\item $\sum_{i=0}^3 {3 \choose i} = 8$

\item $\sum_{i=0}^4 {4 \choose i} = 16$

\item $\sum_{i=0}^8 {8 \choose i} = 256$

\item $\sum_{i=0}^{1000} {1000 \choose i} = 2^{1000}$

\item $\sum_{i=0}^8 {i+1 \choose i} = \sum_{i=0}^8 {i+1 \choose 1} = 9!$

\end{enumerate}
\end{enumerate}
\end{enumerate}

\end{document}

