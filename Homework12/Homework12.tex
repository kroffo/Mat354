\documentclass{scrartcl}
\usepackage{amsmath,amssymb,commath,graphicx,enumerate,listings}
\setkomafont{disposition}{\normalfont\bfseries}

\title{Mat 354}
\subtitle{Homework 12}
\author{Kenny Roffo}
\date{Due November 9, 2015}

\begin{document}
\maketitle

\begin{enumerate}

\item \textbf{Expected Probability}\\
  Consider a continuous random variable $Y$ with (continuous) distribution function $F(y)$. In class it was shown that $E[F(Y)] = \frac{1}{2}$.

  \begin{enumerate}[a)]
  \item Use similar tactics to derive an expression for $E[F^k(Y)]$ where $k>1$.\\

    We note that a the product of a function with its derivative occurrs via the chain rule. Examining this situation, we see

    \begin{align*}
      E[F^k(Y)] &= \int_{all y}F^k(Y)F'(Y)dy\\
      &= \left[\frac{1}{k+1}F^{k+1}(Y)\right]_{all y}\\
      &= \frac{1}{k+1}(1) - \frac{1}{k+1}(0)\\
      &= \frac{1}{k+1}
    \end{align*}

  \item Determine $V[F(Y)]$\\

    By the above expression, we have
    \begin{align*}
      V[F(Y)] &= E(F^2(Y)) - E(F(Y))^2\\
      &= \frac{1}{3} - \frac{1}{2^2}\\
      &=\frac{4}{12} - \frac{3}{12}\\
      &= \frac{1}{12}
    \end{align*}

  \item Do you have a conjecture on the distribution of the random variable $W = F(Y)$?\\

    I may be missing something, but the above appears to apply for any continuous random variable, so I don't see how I would know anything more about it.\\

  \item Suppose instead $Y$ is discrete with distribution function $F(y)$. Is $E[F(Y)] = 1⁄2$? Produce an argument for your assertion.\\

    Nope. Roll a 6-sided die. The probability of any outcome is $\frac{1}{6}$ and the distribution function is $$F(y) = \frac{1}{y} , y\in\{1,2,3,4,5,6\}$$ And of course $$E(F(Y)) = \sum_{y=1}^6F(y)p(y) = \frac{7}{12}$$ Of course, $\frac{7}{12} \ne \frac{1}{2}$.

  \end{enumerate}

\item \textbf{A-maze-ing}\\

  In a lab, a strain of mice are selectively bred for intelligence. And...it turns out that the resulting mice are “smarter than (other) mice”. These mice are timed going through a maze to reach a reward of food. The time (in seconds) required for any mouse is a random variable $Y$ with a density function given by
  \begin{displaymath}
    f(y) = \begin{cases} 
      \frac{2b^2}{y^3} & b \le y \\
      0 & b \nleq y
    \end{cases}
  \end{displaymath}
  where $b$ is the minimum time needed to traverse the maze.

  \begin{enumerate}[a)]
  \item Determine the expected time required.\\

    \begin{align*}
      E(y) &= \lim_{a\rightarrow\infty}\int_b^a y\frac{2b^2}{y^3} dy\\
      &= \lim_{a\rightarrow\infty}\left[-\frac{2b^2}{y}\right]_b^a\\
      &= \lim_{a\rightarrow\infty}\left(-\frac{2b^2}{a}\right) + 2b\\
      &= 2b
    \end{align*}

    The expected time for a mouse to traverse the maze is twice the minimum time required.\\

  \item Which mice have the larger expected time: the intelligent variety of this exercise, or the mice of Exercise 4.15? Explain.\\

    First we calculate the expected time for the mice from Exercise 4.15:

    \begin{align*}
      E(y) &= \lim_{a\rightarrow\infty}\int_b^\infty y\frac{b}{y^2} dy\\
      &= \lim_{a\rightarrow\infty}\left[b\ln(y)\right]_b^a\\
    \end{align*}

    This limit does not exist, which means $E(y)$ does not exist. 

  \end{enumerate}

\item \textbf{(Finally) A Minimum}\\

  A random variable $Y$ (the time until the first occurrence of “something”) has exponential distribution with parameter 2.
  \begin{displaymath}
    F(y) = \begin{cases} 
      0 & y < 0 \\
      1 - e^{-\frac{y}{2}} & y \ge 0
    \end{cases}
    \text{\hspace{0.3in}}
    f(y) = \begin{cases} 
      0 & y < 0\\
      \frac{e^{-\frac{y}{2}}}{2} & y \ge 0
    \end{cases}
    \text{\hspace{0.3in}}
    E(Y) = 2
    \text{\hspace{0.3in}}
    V(Y) = 4
  \end{displaymath}
  There are 5 independent runs of the experiment that result in the random variable $Y$. Let $L$ denote the minimum result for the 5 runs. Determine the following.

  \begin{enumerate}[a)]
  \item The distribution function for $L: F_l(l) = P(L \le l)$. (When handling the minimum, your first step is to attack the probability of the complementary event. Start with $P_l(L > l)$.)\\

    It is apparent that $F_l(L) = 1 - P_l(L > l)$ and we see
    \begin{align*}
      1 - P_l(L > l) &= 1 - P(Y > l)^5\\
      &= 1 - (1-F(l))^5
    \end{align*}

    Thus we have distribution function
    \begin{displaymath}
    F_l(l) = \begin{cases}
      0 & l < 0 \\
      1 - e^{-5l/2} & 0 \le l
    \end{cases}
    \end{displaymath}

  \item The density for $L$.\\

    \begin{align*}
      \od{}{l}(1 - (1 - F(l))^5) &= 5(1 - F(l))^4F'(l)\\
      &= 5(1 - F(l))^4f(l)\\
    \end{align*}

    So the density function is given by
    \begin{displaymath}
    f_l(l) = \begin{cases}
      0 & l < 0 \\
      \frac{5}{2}e^{-5l/2} & 0 \le l
    \end{cases}
    \end{displaymath}

  \item The expected value and variance of $L$.\\

    The following integrals were calculated using wolframalpha.com.
    \begin{align*}
      E(Y) &= \int_0^\infty l\left(\frac{5}{2}e^{-5l/2}\right) dl\\
      &= \frac{2}{5} = 0.4
    \end{align*}

    \begin{align*}
      E(Y^2) &= \int_0^\infty l^2\left(\frac{5}{2}e^{-5l/2}\right) dl\\
      &= \frac{8}{25}
    \end{align*}

    Thus the variance is $V(Y) = \frac{4}{25} = 0.16$

  \item Obtain values for the 0.005 and 0.995 quantiles of the distribution of the minimum. The probability is 0.99 that the minimum falls between these two values.\\

    The expression for the $p^{th}$ quantile is found by solving $F(l)=p$ for $l$:
    \begin{align*}
      &1 - e^{-5l/2} = p\\
      \implies&l = -\frac{2}{5}\ln(1 - p)
    \end{align*}
    
    This implies $$\phi_{0.005} = 0.002005 \text{\hspace{1in}} \phi_{0.995} = 2.119327$$
  \end{enumerate}
\end{enumerate}
\end{document}

