\documentclass{scrartcl}
\usepackage{amsmath,amssymb,commath,graphicx,enumerate,listings}
\setkomafont{disposition}{\normalfont\bfseries}

\title{Mat 354}
\subtitle{Homework 11}
\author{Kenny Roffo}
\date{Due November 4, 2015}

\begin{document}
\maketitle

\begin{enumerate}

\item \textbf{A-maze-ing}\\
In a lab, a strain of mice are selectively bred for intelligence. And...it turns out that the resulting mice are “smarter than (other) mice”. These mice are timed going through a maze to reach a reward of food. The time (in seconds) required for any mouse is a random variable $Y$ with a density function given by 
\begin{displaymath}
f(y) = \begin{cases} 
  \frac{2b^2}{y^3} & b \le y \\
  0 & b \nleq y
\end{cases}
\end{displaymath}
where $b$ is the minimum time needed to traverse the maze.
\begin{enumerate}[a)]
  \item Find $F(y)$. Define it over the entire set of real numbers.\\

    $F(y)$ is found by integrating the density function. We see
    \begin{align*}
      \int_b^y\frac{2b^2}{y^3}dy &= \int_b^y2b^2y^{-3}dy\\
      &= \left[-b^2y^{-2}\right]_b^y\\
      &= 1 - b^2y^{-2}
    \end{align*}
    Thus we have
    \begin{displaymath}
      F(y) = 
      \begin{cases}
        0 & y < b\\
        1 - \frac{b^2}{y^2} & b \le y\\
      \end{cases}
    \end{displaymath}\pagebreak

  \item If $c > 0$ is a positive constant, determine $P(Y > b + c)$.\\
    No words necessary:
    \begin{align*}
      P(Y>b+c) &= 1 - P(Y<b+c)\\
      &= 1 - F(b+c)\\
      &= 1 - (1 - \frac{b^2}{(b+c)^2}\\
      &= \frac{b^2}{(b+c)^2}
    \end{align*}

  \item If $d > c > 0$ are positive constants, find $P(Y > b + d | Y > b + c)$.\\
    Given $Y > b+c$ we can treat $b+c$ as the minimum. Since $d>c$, we know $d-c>0$, so we this probability simplifies greatly. Just note that treating $b+c$ as the minimum means we must use $b+c$ wherever $b$ occurrs in the distribution function. That being said, we can begin the calculation:
\begin{align*}
  P(Y>b+d | Y>b+c) &= P(Y>(b+c)+(d-c))\\
  &= 1 - F(d-c)\\
  &= 1 - (1 - \frac{(b+c)^2}{((b+c)+(d-c))^2}\\
  &= \frac{(b+c)^2}{(b+d)^2}
\end{align*}

\item Determine the $2.5^{th}$ and $97.5^{th}$ percentiles of the times. There is a 95\% probability that a randomly selected mouse’s time falls between these two values. (In other words: 95\% of these mice have times that fall between these two values.)\\
For the $2.5^{th}$ percentile we have
\begin{align*}
  &F(y) = 0.025\\
  \implies& 1 - \frac{b^2}{y^2} = 0.025\\
  \implies& 0.975 = \frac{b^2}{y^2}\\
  \implies& y^2 = \frac{b^2}{0.975}\\
  \implies& y = \frac{b}{\sqrt{0.975}}
\end{align*}
and for the $97.5^{th}$ we've got
\begin{align*}
  &F(y) = 0.975\\
  \implies& 1 - \frac{b^2}{y^2} = 0.975\\
  \implies& 0.025 = \frac{b^2}{y^2}\\
  \implies& y^2 = \frac{b^2}{0.025}\\
  \implies& y = \frac{b}{\sqrt{0.025}}
\end{align*}
Thus the percentiles are given by $$ \phi_{2.5} = \frac{b}{\sqrt{0.975}} \text{\hspace{0.5in} and \hspace{0.5in}} \phi_{97.5} = \frac{b}{\sqrt{0.025}}$$\\

\item State a general expression for finding the $p^{th}$ quantile ($100p^{th}$ percentile) of $Y$; a formula in terms of $p$.\\

That's easy:
$$\phi_{100p} = \frac{b}{\sqrt{1-p}}$$

\item Which mouse is more likely to take more than twice the minimum time: the intelligent mice of this exercise, or the dumb mice of Exercise 4.15? Produce probabilities supporting your response.\\

By the same process as in part a we get that the distribution function for the mice in Exercise 4.15 is $$F(y) = 1 - \frac{b}{y}$$ and also we see that $P(Y>2b) = 1 - F(2b)$. Pluggin in for the distribution functions for this exercise and 4.15 respectively, we see
$$ 1 - \frac{b^2}{(2b)^2} = \frac{3}{4} \text{\hspace{0.5in} and \hspace{0.5in}} 1 - \frac{b}{2b} = \frac{1}{2}$$
Thus the probability is higher that the mice from this exercise will take at least twice as long as the minimum time.
\end{enumerate}

\item \textbf{Max Again}\\
A random variable $Y$ has distribution function
\begin{displaymath}
  F(y) = \begin{cases}
    0 & y < 0\\
    \frac{y(y+1)}{110} & 0 \le y \le 10\\
    1 & 10 < y
  \end{cases}
\end{displaymath}
This was covered in class. The density and some expectations were obtained:
$$E(Y) = 215 / 33 = 6.5152 \text{\hspace{0.5in} and \hspace{0.5in}} V(Y) = 6575 / 1089 = 6.0376$$
Imagine 5 independent runs of the experiment that results in this variable. Let $M$ denote the maximum result for the 5 runs. Determine the following.
\begin{enumerate}[a)]
  \item The distribution function for $M$: $F_M(m) = P_M(M\le m)$ (Nothing has changed from earlier problems of this sort. If the maximum is no greater than $m$, what can you say about each of the 5 outcomes ($Y$s)? Use the independence to obtain an expression for this probability.)\\

    The probability that the maximum of the five trials is a number m, $P(M \le m)$, is the probability that each trial yields an outcome $\le m$. Since all trials are independent, this implies $$F_M(m) = P_M(M\le m) = P(Y\le m)^5 = F(m)^5 = \left(\frac{y(y+1)}{110}\right)^5$$ Note that this is only for choices of $m$ between 0 and 10. Thus truly, we have
\begin{displaymath}
  F_M(m) = 
  \begin{cases}
    0 & m < 0\\
    \left(\frac{y(y+1)}{110}\right)^5 & 0 \le m \le 10\\
    1 & 10 < m
  \end{cases}
\end{displaymath}

\item The density for $M$.\\

The density is found by simply taking the derivative of the distribution, and slapping 0 on the ends past the bounds of possible values:
\begin{align*}
  f_M(m) &= \od{}{m}\left[\left(\frac{m(m+1)}{110}\right)^5\right]\\
  &= 5\left(\frac{m(m+1)}{110}\right)^4\left(\frac{2m+1}{110}\right)
\end{align*}
So
\begin{displaymath}
  f_M(m) = 
  \begin{cases}
    0 & m < 0\\
    5\left(\frac{m(m+1)}{110}\right)^4\left(\frac{2m+1}{110}\right) & 0 \le m \le 10\\
    0 & 10 < m
  \end{cases}
\end{displaymath}

\item The expected value of $M$. (You’ll probably want to offload the integration to technology. It’s easy but tedious.) Clearly state the integral; tell what tools you used to evaluate it; and then present the solution.\\

The expected value is given by $$E(M) = \int_{-\infty}^{\infty}mf_M(m)dm$$ Thus we have
\begin{align*}
  E(M) &= \int_{-\infty}^0 0 dm + \int_0^{10}m\cdot5\left(\frac{m(m+1)}{110}\right)^4\left(\frac{2m+1}{110}\right)dm + \int_{10}^{\infty}0dm\\
  &= 0 + \int_0^{10}m\cdot5\left(\frac{m(m+1)}{110}\right)^4\left(\frac{2m+1}{110}\right)dm + 0\\
  &= 9.0479
\end{align*}
The calculation of this integral was done using Wolfram Alpha.

\item The standard deviation of M.\\
  We find $E(M^2) = 82.619$ by the same process as part c. Now we find the variance
$$V(M) = E(M^2) - E(M)^2 = 0.7545$$
and take the square root for the standard deviation
$$\sigma(M) = \sqrt{V(M)} = 0.8686$$

\item The $5^{th}$ percentile of the distribution of $M$.\\
\begin{align*}
  & F(m) = 0.05 = \left(\frac{m(m+1)}{110}\right)^5\\
  \implies& 60.42 = m(m+1)\\
  \implies m^2 + m - 60.42 = 0\\
  \implies m\in\{-8.2892, 7.2892\}
\end{align*}
Since if $m$ were negative then 0\% of the data would be below it, $m$ must be 7.2892. Thus the $5^{th}$ percentile of the distribution is 7.2892.

\item (Extra) The “obvious” follow up is to ask for parts a – e when there are $n$ (not necessarily 5) independent runs. And then look at what happens as $n\rightarrow\infty$.\\

Since this is extra I won't go into any gory details, but basically, as $n$ grows larger we see\\

a) - The likelihood that all trials are below a value decreases, thus as $n$ gets really big, the distribution function changes to stay close to 0 until it gets close to the maximum $m$ value, which is 10.\\

b) - The density function behaves similaryly to the distribution function.\\

c) - The expected value approaches 10.\\

d) The standard deviation approaches 0.\\

e) The $5^{th}$ percentile approaches 10.
\end{enumerate}
\end{enumerate}
\end{document}

